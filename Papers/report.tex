\documentclass{article}
\usepackage[utf8]{inputenc}
\usepackage{amssymb}
\usepackage{cite}


\title{Rethinking OR with RL}
\author{Maxime Cauté}

\begin{document}

\section{Introduction}

\section{Related Work}


DaRPs have been extensively studied since its first formalization in 1986 by Jaw et al.\cite{jaw1986heuristic}, due to their application to industrial concerns.
The first approaches have been focused on OR resolutions.
An extensive 2018 review can be found in \cite{ho2018survey}.
A deterministic, exact approach was first presented by Cordeau in 2006 \cite{cordeau2006branch}.
The proposed algorithm relied on branch-and-cut methods.
However, exact method suffer from the NP-Hardness of DaRP: their computation time is exponential to the entry size!

Many modern OR approches to tackle this problem are therefore now based on heuristics.

In 2019, Claudia Bongiovanni et al. \cite{bongiovannilearning} proposed the policy which we will consider as 'optimal'.
It relies on a two-phase insertion algorithm:
the requests are, if possible, first assigned to the vehicle best able to respond to it through limited schedule reorganization (e.g. segments shifting);
after a certain amount of new request, an optimization process is then started with intra- and extra-route modifications, for previously approved requests only.
This optimization process notably employs Large Neighborhood Search.
%This method achieved

Reinforcement Learning has also been considered as another solution method for this very problem.
This prospect has only been recently explored, hence a limited amount of related works.

Its potential was yet shown in 2015 by Vinyals et al. \cite{vinyals2015pointer},
who achieved remarkable accuracy in discovering the optimal policy for several combinatorial problems through deep neural networks.
Their architecture, called Pointer Network, adapts Long Short-Term Memories Recurrent Neural Networks for variable input size.

In the field of DaRP application, in 2019, Al-Abbasi et al. \cite{al2019deeppool} considered a deep RL, model-free approach.
They use convolutional, deep Q-networks in order to learn optimal fleet dispatch policies
with regards to both customer (ride time) and company (resource use) points of view.
A notable feature is the use of a distributed decision-making, as
the vehicles learn individually from the environment.
The framework is further exploiting customer statistics, trying to predict further demand.
These two key points lead the authors to better results than frameworks omitting them.

The methods considered above try to compare policies learnt from the environment to expert-given, optimal policies.
In 2018, Stocco and Alahi considered a novel approach for DaRP,
which rather aimed at directly learning the expert policy.
To this extent, they used supervised reinforcement learning through a deep neural network.
As the input was turned into an image, %LeNet ?
the considered architecture consisted in 3 successive, batch normalized, linearly rectified convolutional layers of size 5x5,
followed by 3 similar deconvolutional ones.
Compared to nearest neighbor policy, they report a promising accuracy around $75 \%$,
image projection being likely responsible for a $25\%$ information loss.

%%%%%%%%%%%
%Cordeau also proposed earlier, in 2003, a Tabu Search method for approximate results \cite{TODO}.

%Genetic Algorithm were also considered, first in 2007 by Jorgensen et al. \cite{TODO}.

%Variable Neighborhood search was adapted in 2009 to DaRP by Parragh et al. \cite{TODO}.


%Basic heuristic may also be relevant in the case of dynamic DaRPs, due to the need for fast model-building.
%Elaborate.

%Approximate solutions were also considered.
%In 2010, Gupta et al. \cite{TODO} proposed a $\mathcal{O}(\alpha log^2 n )-approximation algorithm.$

%An approximate solution for the dynamic DaRP was proposed in 2014 by Maalouf et al \cite{TODO}.

%Mention Bongiovanni
%%%%%%%%%%%%%%%%%%%%

\bibliographystyle{unsrt}
\bibliography{biblio}

\end{document}
