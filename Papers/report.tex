\documentclass{article}
\usepackage[utf8]{inputenc}
\usepackage{amssymb}

\title{Rethinking OR with RL}
\author{Maxime Cauté}

\begin{document}

\section{Introduction}

\section{Related Work}


DaRPs have been extensively studied since its first formalization in 1986 by Jaw et al.\cite{TODO}, due to their application to industrial concerns.
The first approaches have been focused on OR resolutions.
An extensive 2018 review can be found in \cite{Ho2018}.
Reinforcement Learning has later been considered as another solution method for this problem.

A deterministic, exact approach was first (?) presented by Cordeau in 2006 \cite{TODO}.
The proposed algorithm relied on branch-and-cut methods.
However, exact method suffer from the NP-Hardness of DaRP: their computation time is exponential to the entry size!
Heuristic approaches have therefore been numerous.

%%%%%%%%%%%
%Cordeau also proposed earlier, in 2003, a Tabu Search method for approximate results \cite{TODO}.

%Genetic Algorithm were also considered, first in 2007 by Jorgensen et al. \cite{TODO}.

%Variable Neighborhood search was adapted in 2009 to DaRP by Parragh et al. \cite{TODO}.


%Basic heuristic may also be relevant in the case of dynamic DaRPs, due to the need for fast model-building.
%Elaborate.

%Approximate solutions were also considered.
%In 2010, Gupta et al. \cite{TODO} proposed a $\mathcal{O}(\alpha log^2 n )-approximation algorithm.$

%An approximate solution for the dynamic DaRP was proposed in 2014 by Maalouf et al \cite{TODO}.

%Mention Bongiovanni
%%%%%%%%%%%%%%%%%%%%

A deep RL approach was recently considered by Al-Abbasi et al. \cite{TODO}

%Vinyals et al. (2015)



\end{document}
